\section{Molecular Simulations}
\subsection{Classical Molecular Dynamics}
Molecular Dynamics utilizes computers to simulate systems of molecules in order to study how structure changes
with time, and the specific ways that different atomic and molecular species interact. This is done
by first creating a model for the Hamiltonian of the system -- a "force-field". This consists firstly of the terms for how atoms
bond to each other, how these bonds move and stretch, and how they rotate around each other; classical models
usually use harmonic potentials for bond and angle stretching and bending, and dihedrals 
are described using periodic functions~\cite{gromacsmanual}.
In addition to the bonded terms, we have the non-bonded terms — energy from 
electrostatic interactions, typically described by Coulomb's law, and dispersion 
interactions (Van der Waals, or VdW) which arise from instantaneous dipole–induced dipole effects~\cite{gromacsmanual}. These are 
most often modeled using a Lennard–Jones (LJ) potential~\cite{Jones:1924,gromacsmanual}, though other forms are 
also common, such as the Buckingham (exp–6) potential~\cite{Buckingham:1938,gromacsmanual}, which replaces the steep 
\(r^{-12}\) repulsive wall with a short-range exponential. In addition to these two, there are many other
functional forms used to describe VdW and dispersion interactions, but they are far less common. An example of this Hamiltonian
can be seen here:
\begin{align}
E_{\mathrm{total}} &= E_{\mathrm{bonded}} + E_{\mathrm{nonbonded}} \\
E_{\mathrm{bonded}} &= \sum_{\text{bonds}} k_r (r - r_0)^2
+ \sum_{\text{angles}} k_\theta (\theta - \theta_0)^2
+ \sum_{\text{dihedrals}} V_n \left[ 1 + \cos\left( n\phi - \gamma \right) \right] \\
E_{\mathrm{nonbonded}} &= \sum_{i<j} \left[ 4\epsilon_{ij} \left( \left( \frac{\sigma_{ij}}{r_{ij}} \right)^{12}
- \left( \frac{\sigma_{ij}}{r_{ij}} \right)^{6} \right)
+ \frac{q_i q_j}{4\pi\epsilon_0 r_{ij}} \right]
\end{align}

By using this set of terms one can compute the potential energy of a particular configuration of particles. 
The forces obtained from the potential energy are used in Newton’s equations of motion, 
which are numerically integrated — most often with algorithms such as the velocity-Verlet method — 
to update positions and velocities at each time step~\cite{gromacsmanual}.
The energy of the new configuration
can be computed, and the simulation continues. Thus, the careful development and improvement of a force-field is critical to 
reproduce valid results that can help us understand what is seen in experiments.

\subsection{\emph{Ab initio} calculations and Density Functional Theory}

Do we really want to get into this? Just enough to understand why we are doing this at all... especially since this is our target data! We
can talk briefly about the methods and things, enough to introduce pbe0+vdw...
\section{Comparing Molecular Simulations with Experimental Results}
\subsection{SAXS (and SANS) -- and other properties obtained from density distributions}
Systems of crystals are often studied using diffraction or scattering experiments. 
In the case of lipid bilayers, we often use Small-Angle scattering of x-rays (SAXS) or neutrons (SANS) 
to study the bilayer structure. Lipid bilayers are smectic crystals, and thus give scattering
patterns that look like concentric circles -- these circles are the bilayer form-factor. This form-factor
is the reciprocal-space structure of the bilayer, but phase information of each lobe is lost. 
Thus, a simple reverse-transformation cannot be perfomed.
In order to produce the appropriate density distribution, one can approximate the number density of bilayer parts as gaussian functions, and 
compute the appropriate scattering density from that. This can then be transformed via a cosine transform
into a form-factor. The gaussians are then adjusted
until the resulting form-factor fits the data from the experiment.
\subsection{Electrostatic and dynamic results... such as GC-theory!}

\subsection{Diffusion coefficients}

\subsection{Order parameters}

