Phospholipids are the main component of cell membranes, providing a boundary that separates cells from their environments and organelles from the cytosol. 
This compartmentalization allows cells to maintain distinct solutions within their cells and organelles, including different mixtures of dissolved ions. 
Consequently, these membranes are constantly flanked by complex salt solutions spanning a range of concentrations. To study membrane structure, one cannot neglect the 
contribution of salts in the local solution. Ions perturb bilayer structure via specific interactions with the phospholipids, primarily the headgroups. 
Experimental studies of these systems are somewhat at odds --- scattering experiments with different salts and zwitterionic lipids report no significant change in bilayer structure with 
concentrations as large as 1 M for monovalent ions and at physiologically relevant concentrations for \mgcl{}. Headgroup NMR studies, however, report that the headgroup tilt angle 
is modulated by ion interactions, indicating that adsorption occurs. Simulations also report thickening of lipid bilayers by ions. 
Here I address this discrepancy by applying a novel optimization procedure that uses \emph{ab initio} substitution energies and associated binding geometries of ions from water to lipid headgroups
as targets for model parameter optimization, replacing Lennard–Jones mixing rules. Simulations of \nacl{} agree far better with experiments than the mixing rule results. 
\mgcl{} proves more complicated, and I propose and characterize two sets of parameters that give different results. I characterize the adsorption modes that \na{}, \li{}, \mg{}, and \cl{} take
when adsorbed to a bilayer of 1-palmitoyl-2-oleoyl-sn-glycero-3-phosphatidylcholine (POPC) and describe how these binding modes perturb lipid bilayer structure. Additional experiments 
are needed to validate the \mg{} parameters, and several validation targets are outlined within. 
