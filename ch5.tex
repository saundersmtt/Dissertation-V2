This dissertation describes and addresses the discrepancy between
experimental lipid bilayer structure and atomistic simulations. To
this end, I have proposed a general method to generate optimized
model parameters for simulation models, and applied that to bilayers
in the presence of \nacl{}, \licl{}, and \mgcl{} salts through the
use of explicit Lennard–Jones interaction cross-terms. The method
attempts to implicitly include multibody cooperativity by using the
substitution energy of ions from water to clusters of molecules that
mimic headgroup chemistry as target data. These targets are computed
using a benchmarked \emph{ab initio} DFT method.

In the case of \nacl{} especially, and \licl{} to a lesser extent, we
observe a reduction in the structural changes caused by dissolved
salt as measured in SAXS form factors, producing results that are
more consistent with experimental observations. In the case of
\mgcl{}, we report varied outcomes. Results from both the original
mixing rules and the optimized parameters required a redefinition of
ion adsorption beyond the strict Langmuir criterion of hydration
shell removal. We define three cases of adsorption: \emph{steric},
where ions enter the bilayer hydration boundary without losing water;
\emph{imperfect}, where ions lose at least one hydrating water in
favor of a lipid component; and \emph{perfect}, where the ion has
fully replaced water with lipid components. Using this framework,
\na{} and \li{} primarily display imperfect or perfect adsorption,
while \mg{} remains exclusively in the steric mode.

These results, along with recent work on \mg{} interactions with
phosphate groups, led us to revisit \mg{} cross-terms with a focus on
fully hydrated target clusters. The outcome was two parameter sets:
one allowing partial dehydration of \mg{} (using parameters from
Grotz \etal{}~\cite{grotzparams}), and another keeping \mg{}
completely hydrated (from Li \etal{}~\cite{merzhfe}). We also observe
a significant number of partial ion pairs, which may change how we
evaluate \cl{} interactions with lipid components in the future. The
choice between these parameter sets produces notably different
results, including substantial bilayer thickening in the case of the
2025 parameter set. This aside, the change in bilayer structure follows 
closely with the number of adsorbed charges in the bilayer interface, with 
no such correlation with the steric adsorption mode. This suggests that dehydrated ions are needed 
to produce a perturbation in lipid-bilayer structure; however, the partial ion-pairs present in the 2025 results 
does beg for further investigation, and their role in the alteration of bilayer structure remains an open question.

In each chapter I propose results with direct experimental
connections. SAXS results with \mg{} are still relatively uncommon
and often performed at concentrations close to physiological, which
may not be high enough to resolve structural changes. NMR order
parameters for headgroups, acyl chains, and solvent remain promising
targets for experimental validation. Deuterium NMR, in particular,
could provide constraints that help refine model parameter searches
and distinguish among competing simulation models. Purely
computational data, even at the \emph{ab initio} level, should not
be relied upon alone to construct reliable interaction models.

By focusing on lipid-ion interactions, this work highlights how relatively subtle changes to a classical molecular dynamics force-field 
can lead to macroscopic changes in lipid-bilayer structure. Implicit inclusion of more complex physics can provide a more computationally efficient way to 
improve force-fields without needing additions to the hamiltonian itself, albeit corrections that are not based on any real physical principle. These methods and the categorization 
ion adsorption, offers a roadmap to continue to improve these simulation models.

\bibliography{refs}
\end{document}


