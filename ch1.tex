\section{Phospholipid bilayers}
Phospholipid bilayers are the primary component of the major delimiting boundary between cells and their environment.
These membranes are not static walls so much as liquid-crystalline, rapidly exploring their conformational
space and adjusting to the external and internal cellular environment. Cells, through the 
diversity of chemistry of headgroups and acyl-chain selections, tailor the dynamic structure of their membrane to further
adapt to the environment around them. 
Understanding the way that individual lipid species contribute to the structure of a lipid bilayer is critical to understand
why a cell might choose one lipid or another in their cell membrane. Additionally, because these lipids are exposed to complex
mixtures of ions on either side of the membrane, the interactions with ions in solution cannot be ignored when discussing the bilayer structure. 
Experimentally observed ion behavior at an interface in solution has been explained using a mean-field model nearly 100 years ago~\cite{israelachvilli:2011:intermol}.
However, this ignores specific chemical interactions between lipid molecules and the ions in the solvent, that are critical for understanding
the effect of changing headgroup species in a membrane. 
Computational models can enable researchers to observe the specific ion-lipid interactions are critical for understanding of 
why cells might choose certain lipids to manipulate their membrane structure under varied ion mixtures in their environment. Furthermore, this 
changing membrane structure also affects the behavior of membrane protiens embedded in the bilayer. Thus, this manipulation of the lipid mixture
presents itself as an important regulatory mechanism for cells.

\section{Molecular Simulations}
\subsection{Classical Molecular Dynamics}
Molecular Dynamics utilizes computers to simulate systems of molecules in order to study how structure changes
with time, and the specific ways that different atomic and molecular species interact. This is done
by first creating a model for the Hamiltonian of the system -- a "force-field". This consists firstly of the terms for how atoms
bond to each other, how these bonds move and stretch, and how they rotate around each other; classical models
usually use harmonic potentials for bond and angle stretching and bending, and dihedrals 
are described using periodic functions~\cite{gromacsmanual}.
In addition to the bonded terms, we have the non-bonded terms — energy from 
electrostatic interactions, typically described by Coulomb's law, and dispersion 
interactions (Van der Waals, or VdW) which arise from instantaneous dipole–induced dipole effects~\cite{gromacsmanual}. These are 
most often modeled using a Lennard–Jones (LJ) potential~\cite{Jones:1924,gromacsmanual}, though other forms are 
also common, such as the Buckingham (exp–6) potential~\cite{Buckingham:1938,gromacsmanual}, which replaces the steep 
\(r^{-12}\) repulsive wall with a short-range exponential. In addition to these two, there are many other
functional forms used to describe VdW and dispersion interactions, but they are far less common. An example of this Hamiltonian
can be seen here:
\begin{align}
E_{\mathrm{total}} &= E_{\mathrm{bonded}} + E_{\mathrm{nonbonded}} \\
E_{\mathrm{bonded}} &= \sum_{\text{bonds}} k_r (r - r_0)^2
+ \sum_{\text{angles}} k_\theta (\theta - \theta_0)^2
+ \sum_{\text{dihedrals}} V_n \left[ 1 + \cos\left( n\phi - \gamma \right) \right] \\
E_{\mathrm{nonbonded}} &= \sum_{i<j} \left[ 4\epsilon_{ij} \left( \left( \frac{\sigma_{ij}}{r_{ij}} \right)^{12}
- \left( \frac{\sigma_{ij}}{r_{ij}} \right)^{6} \right)
+ \frac{q_i q_j}{4\pi\epsilon_0 r_{ij}} \right]
\end{align}

By using this set of terms one can compute the potential energy of a particular configuration of particles. 
The forces obtained from the potential energy are used in Newton’s equations of motion, 
which are numerically integrated — most often with algorithms such as the velocity-Verlet method — 
to update positions and velocities at each time step~\cite{gromacsmanual}.
The energy of the new configuration
can be computed, and the simulation continues. Thus, the careful development and improvement of a force-field is critical to 
reproduce valid results that can help us understand what is seen in experiments.

\subsection{Mixing rules, and NB-fix}
Focusing in particular on the non-bonded terms of the hamiltonian, and specifically the Lennard-Jones (LJ) function, we note that there are two
free parameters \sigmaij{} and \epsilonij{} between a particular pair of particles $i$ and $j$:
\begin{equation}
E_{\mathrm{LJ}_{ij}} = 4\epsilon_{ij} \left( \left( \frac{\sigma_{ij}}{r_{ij}} \right)^{12}
- \left( \frac{\sigma_{ij}}{r_{ij}} \right)^{6} \right)
\end{equation}
Where $r_{ij}$ is the distance between the two particles, \sigmaij{} is the characteristic distance of the Van der Waals interaction between the two species,
and \epsilonij{} is the characteristic energy of the interaction.
These parameters are generally determined for a particular species for the case of the self-interaction, or $\sigma_{ii}$ and $\epsilon_{ii}$.
In this case, they represent roughly the atomic radius, and the ``stickyness'' of the particle. These can then be combined with the 
$\sigma_{jj}$ and $\epsilon_{jj}$ from another species by using what are called mixing rules, developed to 
model the interaction radii and energy well-depth for interacting species~\cite{lorentz:1881,berthelot:1898}.
The Lorentz rule for computing the \sigmaij{} is simply the arithmetic mean of the self terms from the two species, and 
the Berthelot rule for computing the \epsilonij{} is the geometric mean of the self terms.
Modern works have often reported that these mixing rules fail
or behave unpredicatbly for more complicated species than the noble gases~\cite{fyta:2012,boda:2008:effects},
and abandonment of the mixing rules for specifically selected cross-terms can be beneficial to improving
force-fields~\cite{baker:2010:accurate,yoo:2012:improved,fyta:2012:ionic,mamatkulov:2013:force,venable:2013,
savelyev:2014:balancing,li:2015:representation,savelyev:2015:competition,jing:2017:study,reif:2017,wineman:2019}. This 
kind of correction is often called a "non-bonded fix" or NB-fix, which is a correction to the cross-terms of the species of interest 
in order to improve the reproduction of some experimental result.
In this dissertation, I expand on the idea of a NB-fix method to improve the modeling of the interactions of ions and phospholipid species.
However, the method proposed and characterize is general, and can be applied to other chemically diverse systems of interest.

\subsubsection{MD methods utilized in this dissertation}
In the following dissertation, unless otherwise specified, I perform all molecular dynamics simulations with the GROMACS software 
package~\cite{abraham:2015,pall:2014,van:2005,lindahl:2001,berendsen:1995,gromacs}.
I use the SPC/E model for all waters~\cite{spce}, and I describe lipid interactions
with the gromos43A1-S3 parameter set developed by our group in previous work~\cite{chiu:2009}.
System temperature is maintained at 300~K using the Nos\`e--Hoover thermostat
with a coupling constant of 0.5~ps~\cite{nose:1983}, and pressure is maintained
at 1~atm with the Parrinello--Rahman semiisotropic barostat using a coupling constant of 1.5~ps~\cite{parrinello:1981}.
All bonds are constrained with the P--LINCS algorithm, which allows the use of a 4~fs integration timestep~\cite{lincs}.
Equations of motion are integrated with the Verlet scheme, and neighbor lists are updated every 2 steps.
Electrostatic interactions are treated with the particle--mesh Ewald (PME) method~\cite{essmann:1995},
using a real--space cutoff of 1.6~nm and reciprocal grids of either
$56 \times 56 \times 224$ or $52 \times 52 \times 240$ cells,
together with 4th order B--spline interpolation.
Van der Waals interactions are calculated with a single cutoff of 1.6~nm.
\subsubsection{Simulation system construction}
All simulations systems unless otherwise noted are constructed by first creating a leaflet of 100 POPC lipids arranged
on a $10 \times 10$ grid with sufficient spacing to avoid chain overlap.
This leaflet is then reflected along the $z$--axis to generate the second
leaflet, producing a bilayer of 200 lipids. A solvent block is added by
placing 60,000 water molecules on a three--dimensional grid with excess spacing,
and a subset of water molecules is randomly replaced with ions to achieve a
starting concentration of 200~mM (see Table~\ref{tabch3:ions}).

Energy minimization is performed using the steepest--descents algorithm with a
force tolerance of 50~kJ~mol$^{-1}$~nm$^{-1}$. Electrostatic interactions are
treated with the particle--mesh Ewald (PME) method~\cite{essmann:1995}, using a
real--space cutoff of 1.6~nm, reciprocal grid spacing of 0.12~nm$^{-1}$, and
6th order B--spline interpolation. Van der Waals interactions are calculated
with a single cutoff of 1.6~nm. Neighbor searching is performed every 2 steps.

Following energy minimization, I relax the system with a constant--pressure
simulation at 290~K for 200~ps. Simulation systems are then annealled by heating the system above the
simulation run temperature, and then cooling down to the production temperature following the
specific procedure outlined in each following chapter.

\subsection{\emph{Ab initio} calculations and Density Functional Theory}
Classical force-fields are effective for simulating large-systems of particles where an explicit description of 
the electronic behavior is not critical to the understanding of how the particles behave -- while this is a major 
approximation, it turns out to often be useful in studying biological structures like lipid membranes~\cite{berkowitz:2019}.
However, when preparing these classical models, we need to refer to more complicated theory in the form 
of quantum mechanical calculations to set terms like the 
charge per particle in a molecule and to compare to the molecular structure predicted by our bonded-parameters.
\emph{Ab initio} quantum mechanical calculations include chemistry by explicitly describing the behavior of electrons to varied 
levels of accuracy depending on the kind of functional form used, and then the basis vectors used to describe the space.
Pertubation theory methods include electron correlation, which can be useful for assigning electrons to parts of a molecule. 
However, these methods quickly become expensive for systems with many electrons. Density functional theory, which treats the electrons 
with a mean-field approximation is often more computationally efficient and can thus be applied to larger systems.

\emph{Ab initio} methods are used in the works included in this dissertation to generate target data for 
model parameter optimization from geometry
optimizations and substitution energies of small ion–ligand clusters
using density functional theory (DFT) as implemented in
FHI-aims~\cite{fhiaims}. Initial cluster geometries are produced with
the molecular mechanics force field developed in our previous
work~\cite{kruczek:2017,saunders:2019} and then optimized at the DFT
level. I use the PBE0 functional~\cite{perdew:1996:generalized,adamo:1999:toward}
with Tkatchenko–Scheffler dispersion corrections~\cite{tkatchenko:2009},
a combination that has been benchmarked for ion–ligand clusters and shown
to reproduce experimental data and high-level quantum methods across a
range of chemistries~\cite{wineman:2019,wineman:2020:transferable,wineman:2020:improved}.
Geometry optimizations are first performed with the ``light'' basis set
and then refined with the ``really tight'' basis set provided in
FHI-aims. Optimizations are converged to 0.005~eV/\AA{} in forces and
$10^{-6}$~eV in total energy.  

Energies of optimized clusters are then used to compute substitution
energies, defined as the energy of exchanging water molecules
with headgroup analog ligands. The specific cluster sets and substitution
energy targets used for parameterization are described in the relevant
chapters.

\section{Comparing Molecular Simulations with Experimental Results}
\subsection{SAXS (and SANS) -- and other properties obtained from density distributions}
Systems of crystals are often studied using diffraction or scattering experiments. 
In the case of lipid bilayers, we often use Small-Angle scattering of x-rays (SAXS) or neutrons (SANS) 
to study the bilayer structure. Lipid bilayers are smectic crystals, and thus give scattering
patterns that look like concentric circles -- these circles are the bilayer form-factor~\cite{nagle:2000}. This form-factor
is the reciprocal-space structure of the bilayer, but phase information of each lobe is lost. 
Thus, a simple reverse-transformation cannot be perfomed.
In order to produce the appropriate density distribution, one can approximate the number density of bilayer parts as gaussian functions, and 
compute the appropriate scattering density from that~\cite{nagle:2000,fogarty:2015}. This can then be transformed via a cosine transform
into a form-factor. The gaussians are then adjusted
until the resulting continuous form-factor fits the data from the experiment~\cite{nagle:2000,fogarty:2015}.
In simulations, we can compute the number density histograms of the system components
directly from the trajectories, and thus the electron densities directly. Once we have the electron density of our system
per nanosecond, we determine the lipid bilayer centerpoint by locating the minimum at the center of the 
histogram. We symmetrize the histogram around the minimum at the center, and then take an average over many nanoseconds.
We subtract the bulk solvent value of the electron density from the average histogram, and then use a cosine transform
to obtain the bilayer SAXS form-factor. This can be compared with the experimental data directly.
We also use the electron density to directly obtain the peak-to-peak distance of the bilayer, \dhh{}, as one measure
of the bilayer thickness.
\subsection{Number density histograms}
The number density histograms are also used to compute the various bilayer thicknesses via probability densities
of the various system components. We compute 
\subsection{Lipid component volumes}
Phospholipid headgroup and chain occupied volumes are computed using the method of Petrache \etal{} for computing the
volume of immiscible liquids~\cite{petrache:1997}.
In the work outlined in this dissertation, we identify lipid chains are as starting 
at the first carbon attached to the lipid chain carbonyl oxygen, including the oxygen.
The atom groups not part of the lipid
chains are partitioned into the headgroup volume. 
The number--density of these component groups along with that of the solvent are taken 
and used to optimize the objective function:
\begin{equation}
    \label{eq:volumeobj}
    \Omega(v_i)=\sum^{\rho_s}_{z_j}(1-\sum^{N_{\text{Groups}}}_{i=1}{(\rho_i(z_j)v_i)^2})\text{,}
\end{equation}
In the equation above, $\rho_i(z_j)$ is the number density of the $i$ component in the
$z_j$ slice of the box and $v_i$ is the corresponding component volume. 
The component volumes are then multiplied by the corresponding
number of particles per molecule per group -- 32 for the chain
particles, and 20 for the headgroup. 
This gives us the total volume per molecule for each group, defined as \Vh{} and \Vc{}. These are then
added together to make the total \Vl{}.
\subsection{Deuterium Order parameters}
\subsubsection{Acyl-chain CD order parameters}

\subsubsection{D2O Water orientational order parameters}

