\section{Molecular Simulations}
\subsection{Classical Molecular Dynamics}
Molecular Dynamics utilizes computers to simulate systems of molecules in order to study how structure changes
with time, and the specific ways that different atomic and molecular species interact. This is done
by first creating a model for the Hamiltonian of the system -- a "force-field". This consists firstly of the terms for how atoms
bond to each other, how these bonds move and stretch, and how they rotate around each other; classical models
usually use harmonic potentials for bond and angle stretching and bending, and dihedrals 
are described using periodic functions~\cite{gromacsmanual}.
In addition to the bonded terms, we have the non-bonded terms — energy from 
electrostatic interactions, typically described by Coulomb's law, and dispersion 
interactions (Van der Waals, or VdW) which arise from instantaneous dipole–induced dipole effects~\cite{gromacsmanual}. These are 
most often modeled using a Lennard–Jones (LJ) potential~\cite{Jones:1924,gromacsmanual}, though other forms are 
also common, such as the Buckingham (exp–6) potential~\cite{Buckingham:1938,gromacsmanual}, which replaces the steep 
\(r^{-12}\) repulsive wall with a short-range exponential. In addition to these two, there are many other
functional forms used to describe VdW and dispersion interactions, but they are far less common. An example of this Hamiltonian
can be seen here:
\begin{align}
E_{\mathrm{total}} &= E_{\mathrm{bonded}} + E_{\mathrm{nonbonded}} \\
E_{\mathrm{bonded}} &= \sum_{\text{bonds}} k_r (r - r_0)^2
+ \sum_{\text{angles}} k_\theta (\theta - \theta_0)^2
+ \sum_{\text{dihedrals}} V_n \left[ 1 + \cos\left( n\phi - \gamma \right) \right] \\
E_{\mathrm{nonbonded}} &= \sum_{i<j} \left[ 4\epsilon_{ij} \left( \left( \frac{\sigma_{ij}}{r_{ij}} \right)^{12}
- \left( \frac{\sigma_{ij}}{r_{ij}} \right)^{6} \right)
+ \frac{q_i q_j}{4\pi\epsilon_0 r_{ij}} \right]
\end{align}

By using this set of terms one can compute the potential energy of a particular configuration of particles. 
The forces obtained from the potential energy are used in Newton’s equations of motion, 
which are numerically integrated — most often with algorithms such as the velocity-Verlet method — 
to update positions and velocities at each time step~\cite{gromacsmanual}.
The energy of the new configuration
can be computed, and the simulation continues. Thus, the careful development and improvement of a force-field is critical to 
reproduce valid results that can help us understand what is seen in experiments.
\subsubsection{MD methods utilized in this dissertation}
In the following dissertation, unless otherwise specified, I perform all molecular dynamics simulations with the GROMACS software 
package~\cite{abraham:2015,pall:2014,van:2005,lindahl:2001,berendsen:1995,gromacs}.
I use the SPC/E model for all waters~\cite{spce}, and I describe lipid interactions
with the gromos43A1-S3 parameter set developed by our group in previous work~\cite{chiu:2009}.
System temperature is maintained at 300~K using the Nos\`e--Hoover thermostat
with a coupling constant of 0.5~ps~\cite{nose:1983}, and pressure is maintained
at 1~atm with the Parrinello--Rahman semiisotropic barostat using a coupling constant of 1.5~ps~\cite{parrinello:1981}.
All bonds are constrained with the P--LINCS algorithm, which allows the use of a 4~fs integration timestep~\cite{lincs}.
Equations of motion are integrated with the Verlet scheme, and neighbor lists are updated every 2 steps.
Electrostatic interactions are treated with the particle--mesh Ewald (PME) method~\cite{essmann:1995},
using a real--space cutoff of 1.6~nm and reciprocal grids of either
$56 \times 56 \times 224$ or $52 \times 52 \times 240$ cells,
together with 4th order B--spline interpolation.
Van der Waals interactions are calculated with a single cutoff of 1.6~nm.
\subsubsection{Simulation system construction}
All simulations systems unless otherwise noted are constructed by first creating a leaflet of 100 POPC lipids arranged
on a $10 \times 10$ grid with sufficient spacing to avoid chain overlap.
This leaflet is then reflected along the $z$--axis to generate the second
leaflet, producing a bilayer of 200 lipids. A solvent block is added by
placing 60,000 water molecules on a three--dimensional grid with excess spacing,
and a subset of water molecules is randomly replaced with ions to achieve a
starting concentration of 200~mM (see Table~\ref{tabch3:ions}).

Energy minimization is performed using the steepest--descents algorithm with a
force tolerance of 50~kJ~mol$^{-1}$~nm$^{-1}$. Electrostatic interactions are
treated with the particle--mesh Ewald (PME) method~\cite{essmann:1995}, using a
real--space cutoff of 1.6~nm, reciprocal grid spacing of 0.12~nm$^{-1}$, and
6th order B--spline interpolation. Van der Waals interactions are calculated
with a single cutoff of 1.6~nm. Neighbor searching is performed every 2 steps.

Following energy minimization, I relax the system with a constant--pressure
simulation at 290~K for 200~ps. Simulation systems are then annealled by heating the system above the
simulation run temperature, and then cooling down to the production temperature following the
specific procedure outlined in each following chapter.

\subsection{\emph{Ab initio} calculations and Density Functional Theory}
Do we really want to get into this? Just enough to understand why we are doing this at all... especially since this is our target data! We
can talk briefly about the methods and things, enough to introduce pbe0+vdw...
\subsubsection{\emph{Ab initio} methods utilized in this work}
\subsection{Quantum Mechanical Calculations}
I generate target data for parameter optimization from geometry
optimizations and substitution energies of small ion–ligand clusters
using density functional theory (DFT) as implemented in
FHI-aims~\cite{fhiaims}. Initial cluster geometries are produced with
the molecular mechanics force field developed in our previous
work~\cite{kruczek:2017,saunders:2019} and then optimized at the DFT
level. I use the PBE0 functional~\cite{perdew:1996:generalized,adamo:1999:toward}
with Tkatchenko–Scheffler dispersion corrections~\cite{tkatchenko:2009},
a combination that has been benchmarked for ion–ligand clusters and shown
to reproduce experimental data and high-level quantum methods across a
range of chemistries~\cite{wineman:2019,wineman:2020:transferable,wineman:2020:improved}.
Geometry optimizations are first performed with the ``light'' basis set
and then refined with the ``really tight'' basis set provided in
FHI-aims. Optimizations are converged to 0.005~eV/\AA{} in forces and
$10^{-6}$~eV in total energy.  

Energies of optimized clusters are then used to compute substitution
energies, defined as the energy of exchanging water molecules
with headgroup analog ligands. The specific cluster sets and substitution
energy targets used for parameterization are described in the relevant
chapters.

\section{Comparing Molecular Simulations with Experimental Results}
\subsection{SAXS (and SANS) -- and other properties obtained from density distributions}
Systems of crystals are often studied using diffraction or scattering experiments. 
In the case of lipid bilayers, we often use Small-Angle scattering of x-rays (SAXS) or neutrons (SANS) 
to study the bilayer structure. Lipid bilayers are smectic crystals, and thus give scattering
patterns that look like concentric circles -- these circles are the bilayer form-factor. This form-factor
is the reciprocal-space structure of the bilayer, but phase information of each lobe is lost. 
Thus, a simple reverse-transformation cannot be perfomed.
In order to produce the appropriate density distribution, one can approximate the number density of bilayer parts as gaussian functions, and 
compute the appropriate scattering density from that. This can then be transformed via a cosine transform
into a form-factor. The gaussians are then adjusted
until the resulting form-factor fits the data from the experiment.
\subsection{Electrostatic and dynamic results... such as GC-theory!}

\subsection{Diffusion coefficients}

\subsection{Order parameters}

