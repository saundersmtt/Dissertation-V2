Phospholipids are the main component of cell membranes, providing a delimiting boundary for cells from their environments or from their organelles from the cytosol. 
This compartmentalization allows cells to maintain quite different solutions within their cells and organelles, including different mixtures dissolved ions. 
Consequently, these membranes are constantly flanked by complex mixtures of salts at varied concentrations. To study membrane structure, one cannot neglect the 
contribution of the salts in the local solution. Ions perturb bilayer structure via specific interaction with the phospholipids, primarily with the headgroups. 
Experimental studies of these systems are somewhat at odds --- scattering experiments with different salts and zwitterionic lipids report no significant change in bilayer structure with 
concentrations as large as 1 M for monovalent ions, and at physiologically relevant concentrations for \mgcl{}. Groups have reported via headgroup NMR that the headgroup tilt angle 
is modulated by ion interaction, and thus there must be some adsorption that occurs. Simulations also report thickening of lipid bilayers by ions. 
Here this discrepancy is addressed by the application of a novel optimization procedure using \emph{ab initio} substitution energies and associated binding geometries of ions from water to lipid headgroups
as targets for model parameter optimization. We apply parameters to replace Lennard–Jones mixing rules. Simulations of \nacl{} are much more in line with experiments than the mixing rule results. 
\mgcl{} proves more complicated, and herein we propose and characterize two sets of parameters that give different results. We characterize the adsorption modes that \na{}, \li{}, \mg{}, and \cl{} take
when adsorbed to a bilayer of 1-palmitoyl-2-oleoyl-sn-glycero-3-phosphatidylcholine (POPC), and characterize how these binding modes result in perturbation of the lipid bilayer structure. Experiments 
are needed to validate the \mg{} parameters, and several validation targets are provided within. 